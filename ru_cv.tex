\documentclass[letterpaper,11pt]{article}

\usepackage[T2A]{fontenc}
\usepackage[utf8]{inputenc}
\usepackage{latexsym}
\usepackage[empty]{fullpage}
\usepackage{titlesec}
\usepackage[usenames,dvipsnames]{color}
\usepackage{verbatim}
\usepackage{enumitem}
\usepackage[hidelinks]{hyperref}
\usepackage{fancyhdr}
\usepackage[russian,english]{babel}
\usepackage{tabularx}
\usepackage{hyphenat}
\usepackage{fontawesome5}
\input{glyphtounicode}

\pagestyle{fancy}
\fancyhf{} % clear all header and footer fields
\fancyfoot{}
\renewcommand{\headrulewidth}{0pt}
\renewcommand{\footrulewidth}{0pt}

\addtolength{\oddsidemargin}{-0.5in}
\addtolength{\evensidemargin}{-0.5in}
\addtolength{\textwidth}{1in}
\addtolength{\topmargin}{-.5in}
\addtolength{\textheight}{1.0in}

\urlstyle{same}

\raggedbottom
\raggedright
\setlength{\tabcolsep}{0in}

\titleformat{\section}{
  \vspace{-4pt}\scshape\raggedright\large
}{}{0em}{}[\color{black}\titlerule \vspace{-5pt}]

\pdfgentounicode=1

% Custom commands
\newcommand{\resumeItem}[1]{
  \item\small{
    {#1 \vspace{-2pt}}
  }
}

\newcommand{\resumeSubheading}[4]{
  \vspace{-2pt}\item
    \begin{tabular*}{0.97\textwidth}[t]{l@{\extracolsep{\fill}}r}
      \textbf{#1} & #2 \\
      \textit{\small#3} & \textit{\small #4} \\
    \end{tabular*}\vspace{-7pt}
}

\newcommand{\resumeSubItem}[1]{\resumeItem{#1}\vspace{-4pt}}

\renewcommand\labelitemii{$\vcenter{\hbox{\tiny$\bullet$}}$}

\newcommand{\resumeSubHeadingListStart}{\begin{itemize}[leftmargin=0.15in, label={}]}
\newcommand{\resumeSubHeadingListEnd}{\end{itemize}}
\newcommand{\resumeItemListStart}{\begin{itemize}}
\newcommand{\resumeItemListEnd}{\end{itemize}\vspace{-5pt}}

%%%%%%  RESUME STARTS HERE  %%%%%%%%%%%%%%%%%%%%%%%%%%%%

\begin{document}

%---------- HEADING ----------
\begin{center}
  \textbf{\Huge \scshape Поляков Игорь} \\ \vspace{3pt}
    \small
    \faAt \hspace{.5pt} \href{mailto:igorpolyakov@protonmail.com}{igorpolyakov@protonmail.com}
    $|$
    \faTelegram \hspace{.5pt} \href{https://t.me/hotorcelexo}{@hotorcelexo}
    $|$
    \faLinkedinIn \hspace{.5pt} \href{https://www.linkedin.com/in/polyakov-igor-b63497133/}{LinkedIn}
    $|$
    \faGithub \hspace{.5pt} \href{https://github.com/IgorPolyakov}{GitHub}
    $|$
    \faMapPin \hspace{.5pt} Санкт-Петербург, Россия
\end{center}

%----------- WORK EXPERIENCE -----------
\section{Опыт работы}
  \vspace{3pt}
  \resumeSubHeadingListStart
    \resumeSubheading
      {Kupibilet}{Санкт-Петербург, Россия}
      {Бэкенд разработчик}{Ноябрь 2018 \textbf{--} Июль 2023}
      \resumeItemListStart
          \resumeItem{Доработки по сервису аккаунта, хранение и обработка персональных данных, данных банковских карт. Стек: Go / MySQL / Iris / GORM.}
          \resumeItem{Доработки по сервису отправки сообщений. Стек: Go / RabbitMQ / Iris. Реализована интернационализация сообщений с использованием Gobuffalo Plush.}
          \resumeItem{Разработка с нуля информационной системы для сервиса "выбор места". Стек: Go / Gin / OpenAPI.}
          \resumeItem{Поддержка и развитие сервиса для работы с банковскими картами, необходимый для соответствия нормам PCI DSS. Стек: Go / Mux / Redis.}
          \resumeItem{Участие в аудите PCI DSS, внедрение инструментов безопасности, таких как Nancy, для безопасной работы с данными пользователей.}
          \resumeItem{Доработки существующего API для бронирования и продажи авиабилетов. Стек: Ruby 2 / MongoDB 3.}
          \resumeItem{Написание и поддержка интеграций с платежными системами (Тинькофф, Checkout.com, YooKassa, Pay2Me). Стек: Ruby / Dry-* / PSQL / Roda / ROM.}
          \resumeItem{Подключение и работа с онлайн-кассой, создание чеков (фискальный документ). Стек: Orange Data / Ruby.}
          \resumeItem{Добавление мониторинга и визуализации работы приложения и инфраструктуры. Стек: Prometheus / Grafana / Elasticsearch / Logstash / Kibana / Sentry.}
          \resumeItem{Внедрение на проекте инструментов и подходов для интернационализации основного API. Стек: Rails-i18n / Lokalise.com / Weblate.org / I18n-tasks.}
          \resumeItem{Решение задач связанных с внедрением continuous integration и continuous delivery (CI/CD), написание сценариев для статического анализа, тестирования и отправки кода на тестовые сервера. Стек: Self-hosted GitLab / Traefik / Docker / Docker-compose / HAProxy.}
          \resumeItem{Поддержка API для генерации лендингов (SEO). Стек: Ruby 2 / Rails 5 / MongoDB 3 / Bootstrap.}
          \resumeItem{Поддержка и развитие парсеров необходимых для бизнеса, таких как парсеры цен, IATA объектов и т.д. Стек: Ruby / Selenium / Curl.}
          \resumeItem{Поддержка и развитие сервиса поисковых подсказок (suggest). Стек: Elixir / Cowboy.}
          \resumeItem{Поддержка внутренних инструментов команды. Стек: Crystal / JIRA-API / GitLab-API / Slack-API.}
        \resumeItemListEnd
    \resumeSubheading
      {Elecard}{Томск, Россия}
      {Инженер-программист}{Август 2017 \textbf{--} Ноябрь 2018}
        \resumeItemListStart
          \resumeItem{Доработки существующего проекта Boro Elecard. Стек: Rails 4 / jQuery}
          \resumeItem{Разработка внутреннего сервиса для генерации и управления цифровыми сертификатами (X.509). Стек: Rails 4 / OpenSSL.}
          \resumeItem{Разработка и поддержка JRPC сервера. Стек: ruby.}
        \resumeItemListEnd
  \resumeSubHeadingListEnd
%----------- EDUCATION -----------
\section{Образование}
  \vspace{3pt}
  \resumeSubHeadingListStart
    \resumeSubheading
      {Томский университет систем управления и радиоэлектроники}{Томск, Россия}
      {Автоматизация технологических процессов и производств; Вычислительных систем}{Сен. 2010 \textbf{--} Янв. 2015}
    \resumeSubheading
      {Институт физики прочности и материаловедения СО РАН}{Томск, Россия}
      {Аспирантура - Информатика и вычислительная техника}{Сен. 2015 \textbf{--}  Янв. 2017}
  \resumeSubHeadingListEnd
%----------- SKILLS -----------
\section{Навыки}
\vspace{2pt}
\resumeSubHeadingListStart
\small{\item{
\textbf{Языки программирования:} Go, Ruby, Elixir, SQL \vspace{3pt}
\textbf{Технологии:} Linux, Git, Docker, PostgreSQL, MongoDB, CI/CD \vspace{3pt}
\textbf{Языки:} Русский (родной), Английский (начальный уровень)
}}
\resumeSubHeadingListEnd

%----------- EDUCATION (Update)-----------
\section{Повышение квалификации, курсы}
\vspace{2pt}
\resumeSubHeadingListStart
\resumeSubItem{\href{https://cert.tough-dev.school/9iEfy3ppz8G5DMLcTsqkxT/ru}{Курс «Асинхронная архитектура»}, Школа Сильных Программистов, 2022}
\resumeSubHeadingListEnd
\end{document}
